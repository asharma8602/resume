%!TEX TS-program = xelatex
%!TEX encoding = UTF-8 Unicode
% Awesome CV LaTeX Template for CV/Resume
%
% This template has been downloaded from:
% https://github.com/posquit0/Awesome-CV
%
% Author:
% Claud D. Park <posquit0.bj@gmail.com>
% http://www.posquit0.com
%
% Template license:
% CC BY-SA 4.0 (https://creativecommons.org/licenses/by-sa/4.0/)
%


\documentclass[10pt, a4paper]{awesome-cv}
\geometry{left=1.4cm, top=.8cm, right=1.4cm, bottom=1.4cm, footskip=.5cm}
\fontdir[fonts/]

% Color for highlights
% Awesome Colors: awesome-emerald, awesome-skyblue, awesome-red, awesome-pink, awesome-orange
%                 awesome-nephritis, awesome-concrete, awesome-darknight
\colorlet{awesome}{awesome-darknight}
% Uncomment if you would like to specify your own color
% \definecolor{awesome}{HTML}{CA63A8}

% Colors for text
% Uncomment if you would like to specify your own color
% \definecolor{darktext}{HTML}{414141}
% \definecolor{text}{HTML}{333333}
% \definecolor{graytext}{HTML}{5D5D5D}
% \definecolor{lighttext}{HTML}{999999}

% Set false if you don't want to highlight section with awesome color
\setbool{acvSectionColorHighlight}{false}

% If you would like to change the social information separator from a pipe (|) to something else
\renewcommand{\acvHeaderSocialSep}{\quad\textbar\quad}


% Available options: circle|rectangle,edge/noedge,left/right
% \photo{./profile.png}
\name{Yash}{Srivastav}
\position{Bachelor of Technology{\enskip\cdotp\enskip}Computer Science and Engineering}
\address{Indian Institute of Technology, Kanpur}
\mobile{(+44) 7909008724}
\email{yash111998@gmail.com}
\homepage{yashsriv.org}
\github{yashsriv}
\linkedin{yashsriv}
% \twitter{@therealyashsriv}
% \quote{``There is no fate but what we make."}

\newcommand{\smallcventry}[6]{\cventry{#1}{#2}{#3}{#4}{#6}}
\newcommand{\specialcvsection}[1]{\cvsection{#1}}




\begin{document}
\makecvheader
\makecvfooter
  {}
  {}
  {\thepage}

\specialcvsection{Educational Qualifications}

\newcommand{\education}[4]{
  & #1 & #2 & &#3 & #4
}
\begin{center}
\begin{tabular}{ | L{0.05cm} l | L{3cm} | L{0.05cm} C{7cm} | r |}
  \hline
  \education{\textbf{Year}}{\textbf{Degree}}{\textbf{Institution(Board)}}{\textbf{CGPA/\%}}\\
  \hline
  \education{July'15 -- June'19}{B.Tech, CSE}{Indian Institute of Technology, Kanpur}{9.1/10.0}\\
  \education{2015}{AISSCE -- XII}{Birla High School, Kolkata (CBSE)}{96.6\%}\\
  \education{2013}{ICSE -- X}{AG Church School, Kolkata (CISCE)}{96.6\%}\\
  \hline
\end{tabular}
\end{center}
\vspace{-4mm}

%%% Local Variables:
%%% mode: latex
%%% TeX-master: "../cv.tex"
%%% TeX-engine: xelatex
%%% End:

\cvsection{Work Experience}
\begin{cventries}

  \cventry
  {Software Engineer}
  {\href{https://www.facebook.com}{Facebook, London}}
  {United Kingdom}
  {August 2019 - Present}
  {
    \begin{cvitems}
    \item University Graduate Recruit
    \item Developed features for \href{https://www.workplace.com}{Workplace by
        Facebook}, one of the fastest growing SaaS products in the world.
    \item Currently working on \href{https://www.messenger.com}{Messenger}
      \ifdefined \ONEPAGE \else
    \item Technologies used: React, Relay, Graphql, Hack
      \fi
    \end{cvitems}
  }

  \cventry
  {Full Stack Developer, Prof. Manindra Agrawal}
  {\href{https://yashsriv.org/2016/10/thoughts.html}{New York Office, IIT Kanpur}}
  {India}
  {May 2016 - May 2019}
  {
    \begin{cvitems}
      \item Summer Intern continued as a volunteer \& peer mentor.
      \item Worked on a scalable web application with an extensive technology
        stack using Postgres, Couchbase, Kafka, etc.
      \item	Implemented Notifications, XSRF \& XSSI Protection and batch
        processing of api requests in the backend API.
      \item Integrated ProseMirror into an existing Angular App
      \item Reimplemented an entire Angular application as per the redux
        architecture for easy extensibility and modularity.
        \ifdefined \ONEPAGE \else
      \item Technologies used: Scala with Akka, Couchbase, Angular with TypeScript, Redux
        \fi
    \end{cvitems}
  }

  \cventry
  {Summer Intern, Prof. Justin Cappos}
  {\href{https://yashsriv.org/2018/07/nyu.html}{New York University}}
  {New York City, USA}
  {May 2018 - July 2018}
  {
    \begin{cvitems}
      \item Worked on setting up an one-shot ansible project to enable
        organizations to easily set up debian package rebuilding infrastructure
        for independent verification of packages.
      \item Implemented three different microservices to orchestrate the entire
        setup and seamlessly display results via an API.
    \ifdefined\ONEPAGE
    \else
        A builder to rebuild the packages, a visualizer to expose all the
        metadata \& buildinfo and a scheduler to schedule building of packages.
        \fi
      \item Interacted with the reproducible-builds community to get feedback
        and design decisions approved.
      \item Added support for ed25519 cryptographic keys to the supply chain security
        framework - \href{https://in-toto.io/}{in-toto}.
        \ifdefined \ONEPAGE \else
      \item Technologies Used: Ansible, Python, Perl, debian build tools
        \fi
    \end{cvitems}
  }

  \cventry
  {OpenPrinting, The Linux Foundation}
  {\href{https://yashsriv.org/2017/08/gsoc.html}{Google Summer of Code}}
  {}
  {Summer 2017}
  {
    \begin{cvitems}
      \item Worked on the Common Printing Dialog Project.
      \item Allows easy integration of any new printing systems into existing
        applications implementing the interface.
      \item Helped design the DBus API and decide its usage.
      \item I worked with dbus and implemented the frontend print dialogs to the
        dbus interface in Libreoffice codebase.
        \ifdefined \ONEPAGE \else
      \item Technologies Used: DBus, C++, CUPS, Google Cloud Printing
        \fi
    \end{cvitems}
  }

  \cventry
  {Full Stack Developer}
  {\href{http://www.localites.com}{Localites}}
  {Freelance}
  {Dec 2018 - Apr 2018}
  {
    \begin{cvitems}
    \item Worked on a web application using the MERN stack.
    \item Rewrote the entire nodejs server in typescript with unit testing in
      jasmine for type-checking and code reliability.
    \item Also rewrote react app into an NgRx based angular app for better code
      organization and structure.
        \ifdefined \ONEPAGE \else
    \item Technologies Used: MongoDB, React, NodeJS, Angular
        \fi
    \end{cvitems}
  }

\end{cventries}
\vspace{-2mm}

%%% Local Variables:
%%% mode: latex
%%% TeX-engine: xetex
%%% TeX-master: "../cv.tex"
%%% End:

%\input{sections/academic.tex}
\cvsection{Skills}
\ifdefined\ONEPAGE
\\
\textbf{Proficient}: Javascript, Hack, C, Golang, Python, \\\vspace{0.3mm}
\textbf{Experienced}: C++, Java, Scala\\\vspace{0.3mm}
\textbf{Exposure}: Haskell, Rust, Dart, Perl, Android\\\vspace{0.3mm}
\textbf{Web}: React, Relay, Angular, Akka, TypeScript, Redux\\\vspace{0.3mm}
\textbf{Utilities}: Shell Utilities, Git, Docker, Ansible, PostgreSQL, MongoDB, OpenCV,
\LaTeX, Vim, Emacs, Vagrant

\else
\begin{cvskills}

  \cvskill
  {Proficient}
  {Javascript, Hack, C, Golang, Python}

  \cvskill
  {Experienced}
  {C++, Java, Scala}

  \cvskill
  {Exposure}
  {Haskell, Rust, Dart, Perl, Android}

  \cvskill
  {Web}
  {Express.js with Node.js, Akka with Scala, JavaScript, TypeScript, React, GraphQL, Angular, Redux}

  \cvskill
  {Utilities}
  {Shell utilities, Git, Docker, Ansible, Postgres,
    MongoDB, OpenCV, \LaTeX, Vim, Emacs, vagrant}

\end{cvskills}
\fi
%%% Local Variables:
%%% mode: latex
%%% End:

%\cvsection{Relevant Courses}

\ifdefined\ONEPAGE

% \textbf{CS:} Introduction to Programming(A$*$), Logic in Computer
% Science, Computer Organization, Data Structures and Algorithms, Computing
% Laboratories - 1(A$*$)
\begin{tabular*}{\textwidth}{l l l l}
  Introduction to Programming(A$*$) & Discrete Mathematics  & Computer
                                                              Organization &
                                                                             Computer Architecture \\
  Data Structures and Algorithms & Probability \& Statistics(A$*$) & Computing
                                                                     Lab
                                                                     - 1(A$*$) &
                                                                                 Computing
                                                                                 Lab
                                                                                 -
                                                                                 2(A$*$) \\
  Compiler Design &  Functional Programming(A$*$) & Computer Systems Security & Computer Networks
\end{tabular*}



% \textbf{Math}: Discrete Math, Probability and Statistics(A$*$)


{\footnotesize
    {A$*$: Grade for exceptional performance}
}

\else
{\fontsize{11pt}{1em}\bodyfontlight\upshape\color{text}
  \begin{tabular*}{\textwidth}{l l l}
    Introduction to Programming(A$*$) & Discrete Mathematics  & Computer Organization \\
    Computer Architecture & Data Structures and Algorithms & Probability \& Statistics(A$*$) \\
    Computing Laboratories - 1(A$*$) & Computing Laboratories - 2(A$*$) & Compiler Design \\
    Functional Programming(A$*$) & Computer Systems Security & Computer Networks
  \end{tabular*}
}
{\fontsize{11pt}{1em}\footerfont\upshape\color{text}
  \entrylocationstyle{A$*$: Grade for exceptional performance}\\
}
\vspace{-0.5cm}

\fi

%%% Local Variables:
%%% mode: latex
%%% TeX-engine: xetex
%%% TeX-master: "../cv"
%%% End:

%\newpage
\cvsection{Projects}

\begin{cventries}

  \cventry
  {Inter IIT Tech Meet}
  {\href{https://github.com/yashsriv/beethoven}{Dashboard Web Extension/App}}
  {\emph{\texttt{\href{https://github.com/yashsriv/beethoven}{github://yashsriv/beethoven}}}}
  {February, 2017}
  {
    \begin{cvitems}
    \item Involved creating a web extension which acted as a user’s home page
      \ifdefined \ONEPAGE and displayed relevant information to a student. \else
      and helped display all the information relevant to a student studying in
      IIT Kanpur in a main Dashboard. (News, Events, Student Search, Share Auto)
      \fi
    \item Server Side was implemented as multiple \textbf{microservices} in
      various languages \ifdefined \ONEPAGE \else (Golang, Python, NodeJS) \fi
      involving IPC via \textbf{JSON RPC}.
    \item Fully \textbf{Dockerized backend} running on a docker-compose setup.
      Written with \textbf{scalability} and speed in mind.
    \item Judged \textbf{1\textsuperscript{st}} among all IITs participating
      in the competition.
    \end{cvitems}
  }

  \smallcventry
  {Course Project}
  {\href{https://github.com/yashsriv/networks-video-stream}{Live Lecture Streaming (websockets + webRTC)}}
  {Computer Networks}
  {Nov'2018}
  {\emph{\texttt{\href{https://github.com/yashsriv/networks-video-stream}{github://yashsriv/networks-video-stream}}}}
  {
    \begin{cvitems}
    \item A live lecture streaming service with live discussion
    \item Used websockets for communicaiton and webRTC for video transmission.
    \end{cvitems}
  }

  \smallcventry
  {Course Project, Computer Architecture}
  {\href{https://github.com/yashsriv/branch-predictor/blob/master/report/main.pdf}{Branch Predictor}}
  {Best Predictor}
  {April'2018}
  {\emph{\texttt{\href{https://github.com/yashsriv/branch-predictor/blob/master/report/main.pdf}{github://yashsriv/branch-predictor}}}}
  {
    \begin{cvitems}
    \item Designed a branch predictor for an intra-class branch prediction
      championship based on the CBP-1 framework in a team of 2.
    \item Created a modified GEHL predictor with an additional loop predictor.
    \item Was adjudged the \textbf{best predictor} amongst all submitted.
    \end{cvitems}
  }

  \smallcventry
  {Course Project}
  {\href{https://github.com/yashsriv/haskell-connect-4}{Connect 4 AI in haskell}}
  {Functional Programming}
  {Jan'2018-April'2018}
  {\emph{\texttt{\href{https://github.com/yashsriv/haskell-connect-4}{github://yashsriv/haskell-connect-4}}}}
  {
    \begin{cvitems}
    \item A GUI based connect 4 AI in haskell.
    \item Had support for various difficulties and the AI was abstracted out in
      order to be able to support any complete knowledge two player game.
    \end{cvitems}
  }

  \smallcventry
  {Course Project, Compiler Design}
  {\href{https://github.com/yashsriv/tango}{tango} \strong{(\emph{golang to x86 assembly})} }
  {}
  {Jan'2018-April'2018}
  {\emph{\texttt{\href{https://github.com/yashsriv/tango}{github://yashsriv/tango}}}}
  {
    \begin{cvitems}
    \item A compiler for go written in go in a team of 3. Compiles from golang
      to x86 assembly.
    \item Supports a subset of the go language including nested pointers, type
      checking, recursion, nested arrays, structs, methods and other common
      programming language features.
    \item Added a new for comprehension syntax as well to golang.
    \end{cvitems}
  }

  \smallcventry
  {Self Project}
  {\href{https://github.com/yashsriv/go-nachos}{go-nachos}}
  {Ported nachOS to golang}
  {Dec'2017}
  {\emph{\texttt{\href{https://github.com/yashsriv/go-nachos}{github://yashsriv/go-nachos}}}}
  {}

  \smallcventry
  {Programming Club}
  {\href{http://pclub.in/project/2016/07/06/smartmirror.html}{Smart Mirror}}
  {Best Applicable Project}
  {Summer'2016}
  {\emph{\texttt{\href{https://github.com/11000011/Smart-Mirror}{github://11000011/Smart-Mirror}}}}
  {
    \begin{cvitems}
    \item Built an \textbf{IoT Mirror} with an RPi and a display fitted with a 75\%
      reflecting mirror.
    \item The mirror had features such as weather forecast, calendar
      and pushbullet notifications of a user (determined via face
      identification).
      \item I was involved in integrating Google Calendar notifications and Face
        Identification using Microsoft’s Project Oxford API.
    \item Received \textbf{Best Applicable Project} amongst all summer projects under
      the Science and Technology Council, IIT Kanpur.
    \end{cvitems}
  }

  \smallcventry
  {Association of Computing Activities}
  {\href{https://github.com/yashsriv/Reversi-Python}{Reversi Game in Python}}
  {IIT Kanpur}
  {2\textsuperscript{nd} Semester}
  {\emph{\texttt{\href{https://github.com/yashsriv/Reversi-Python}{github://yashsriv/Reversi-Python}}}}
  {
    \begin{cvitems}
    \item Developed a Python Application using \textbf{Pygame} for 2 player as well as
      single player Reversi gameplay.
    \item Uses the \textbf{negamax algorithm} with an efficient heuristic check
      for better performance against humans.
    \item Mid Semester project under the Association of Computing Activities (ACA), IIT Kanpur.
    \item Link: \href{https://github.com/yashsriv/Reversi-Python}{github.com/yashsriv/Reversi-Python}
    \end{cvitems}
  }

  \cventry
  {Member, Team Robocon IIT Kanpur\ifdefined\ONEPAGE\else, Prof. Bhaskardas Gupta\fi}
  {ABU Robocon 2016}
  {IIT Kanpur}
  {Oct'2015 - Mar'2016}
  {
    \begin{cvitems}
      \item An autonomous robot, which did not contain a driving actuator had to
        traverse a game field using energy provided to it by another robot in
        form of a non contact force.
      \item Was involved in \textbf{Image Processing} used in the autonomous
        robot for \textbf{color detection} and \textbf{line following}
        \ifdefined \ONEPAGE \else to traverse the arena. \fi
      \item Came \textbf{3\textsuperscript{rd}} out of 105 teams in Nationals at Pune, India.
    \end{cvitems}
  }

  \smallcventry
  {24 Hour Hackathon}
  {Code.Fun.Do}
  {Microsoft India}
  {Sept'2015}
  {Best 5 ideas}
  {
    \begin{cvitems}
    \item Developed an App to help connect teachers and learners based on their
      preference of subjects.
    \item Used cross-platform \textbf{Universal App Platform} for Windows 10
      and a server written in C\#.
    \item Was selected as one of the \textbf{best five ideas}.
    \end{cvitems}
  }

\end{cventries}
\vspace{-2mm}

%%% Local Variables:
%%% mode: latex
%%% TeX-master: "../cv.tex"
%%% TeX-engine: xelatex
%%% End:

\cvsection{Positions of Responsibility}

\begin{itemize}
\item \textbf{Head, Web}, \emph{Antaragni 2017}
\ifdefined\ONEPAGE
\else
  : \\
  Worked on a full MEAN stack application. As part of the Core Team was involved
  in decisions regarding the festival and was responsible for managing the stays
  and travels of all Celebrities and Artists invited to the event.
\fi
\item \textbf{Coordinator}, \emph{Programming Club, IIT Kanpur 2017-18}
\ifdefined\ONEPAGE
\else
  : \\
  Conducted lectures for freshmen and organised competitions. Took the
  initiative of conducting a \textbf{``Winter Camp''} where a select few
  freshmen were introduced to various topics.
  ranging from cryptography to web development.
\fi

\ifdefined\ONEPAGE
\else
\item \textbf{Secretary}, \emph{Programming Club, IIT Kanpur 2016-17}
  : \\
  Helped Conduct and organize various lectures for freshmen as well as developed
  a few web applications under the programming club.
\item \textbf{Senior Executive, Web}, \emph{Antaragni 2016}
  : \\
  Worked on a NodeJS webserver for a college fest. Had a dynamic website
  modifiable easily by non-programmers and supported android app as well with an
  API.
\fi
\end{itemize}
\vspace{-2mm}

\cvsection{Miscellaneous}

\begin{itemize}
  \item Contribute to Open Source projects like pdf.js and thelounge
  \item Among the top 15 teams of India in CSAW 2016, CTF
  \item Teaching Assistant for the course CS251: Computing Laboratories
  \item Expoited and patched the zoobar server as part of Computer Systems
    Security Course
  \item Developed an android app which was a Websocket Client for a
    Websocket Server hosting a multiplayer game
  \item Won Fresher's Science Quiz in inter-hall annual competition
  % \item Mentored 6 students in building a chat application
  %   \ifdefined \ONEPAGE \else
  %   using nodejs and
  %   websockets as an Semester Project
  %   \fi
  %   \vspace{-1mm}
\end{itemize}

% \input{sections/interests.tex}


\end{document}

%%% Local Variables:
%%% mode: latex
%%% TeX-engine: xetex
%%% End:
